\begin{theorem}\label{theo:ame_nizk_priv}
    Let $\Pi', \nizk$ be as above.
    If $\Pi'$ is CPA private, and $\nizk$ satisfies true-simulation extractability for $f(\sattr, \signature, \raccess, \msg, r) = (\sattr, \raccess, \msg)$, then the ME scheme $\Pi$ from construction~\ref{constr:ame_nizk_priv} is CCA-private and preserves its authenticity.
\end{theorem}

\begin{proof}
    We can separately prove the two properties, by using Lemma~\ref{lemma:ame_priv} and Lemma~\ref{lemma:ame_auth_same}

    \begin{lemma}\label{lemma:ame_priv}
        Constructions~\ref{constr:ame_nizk} and \ref{constr:ame_nizk_priv} achieve CCA privacy.
        \begin{proof}
            As before, this proof is almost the same as Lemma ~\ref{lemma:me_priv} (wlog, we only prove the lemma for construction~\ref{constr:ame_nizk}): the only difference is that here we do not need to simulate anymore the generation of the policy keys.
            Let assume that $\Pi$ does not achieve CCA-privacy.
            This implies that there exists a valid $\adversary$ able to win with non-negligible probability $\NISHmatchgamecca{\arranged}_{\Pi, \adversary}(\secpar)$.
            If this is the case, we can build a valid $\adversary'$ to win with non-negligible probability $\NISHmatchgame{\arranged}_{\Pi', \adversary'}(\secpar)$.
            The reduction is the following.

            \begin{enumerate}
                \item (setup) $\adversary'$ receives the master public key $\mpk'$ from the challenger. Then it computes $(\crs,\tpsim,\tpext)\getsr\nizkext_0(\secparam)$, the signatures keys $(\sk, \pk) \getsr \kgen(\secparam)$ and gives $\adversary$ the new master public key $\mpk = (\mpk', \pk, \crs)$.
                \item ($\ora_1$) on input $\sattr$, $\adversary'$ invokes $\ora_1(\sattr)$ to have back $\ek'_\sattr$; moreover, it computes $\signature \getsr \sign(\sk, \sattr)$. Then it gives $\adversary$ the new sender key $\ek_\sattr = (\ek'_\sattr, \signature)$.
                \item ($\ora_2$) on input $(\rattr, \saccess)$, $\adversary'$ invokes $\ora_2(\rattr, \saccess)$ and forwards the received $\ek'_{\rattr, \saccess}$.
                \item (decryption - $\ora_3$) on input $(\rattr, \saccess, (\cipher', \pi))$, it first parses the ciphertext and checks whether the proof $\pi$ is valid or not (in this case simply outputs $\bot$). Then, extracts $(\sattr, \signature, \raccess, \msg)$ using $\nizkext_1(\tpext,\cipher',\pi)$ and if there is a match returns $\msg$, otherwise returns $\bot$.
                \item (challenge) the input tuple is passed to the challenger that computes $\cipher'_b$; $\adversary'$ returns back $(\cipher'_b, \pi)$, where $\pi = \nizksim_1(\tpsim, \cipher'_b)$ is a simulated proof for $\cipher'_b$.
                \item (output) the bit $b'$ is given to the challenger.
            \end{enumerate}
            ~\newline
            The view offered by $\adversary'$ to $\adversary$ is computationally close to the original one: the difference, again, is due to the different mechanism to generate the CRS $\crs$, but this cannot be distinguished by $\adversary$, because of the assumption of f-tSE of $\nizk$.
            Oracles$\ora_2$ is perfectly simulated since it is offered directly by the challenger.
            $\ora_1$ queries are again perfectly simulated because the sender key is honestly generated by the challenger and the signature is valid (and verifiable by $\adversary$).
            By assumption, $\ora_3$ queries are perfectly simulated except with some negligible probability, due to the failure of the extractor $\nizkext_1(\tpext, \cdot, \cdot)$: the correctness is still preserved because $\adversary'$ can extract all the necessary information to successfully decrypt a message if and only if a match occurs.
        \end{proof}
    \end{lemma}

    \begin{lemma}\label{lemma:ame_auth_same}
        Construction~\ref{constr:ame_nizk} preserves its authenticity.
        \begin{proof}
            Let assume $\Pi$ does not preserve its authenticity.
            This implies that there exists a valid $\adversary$ able to win with non negligible probability $\NISHeufgame{\arranged}_{\Pi, \adversary}(\secpar)$ (resp. $\NISHeufgamecca{\arranged}_{\Pi, \adversary}(\secpar)$).
            If this is the case, we can build a valid $\adversary'$ to win with non negligible probability $\NISHeufgame{\arranged}_{\Pi', \adversary'}(\secpar)$ (resp. $\NISHeufgamecca{\arranged}_{\Pi', \adversary'}(\secpar)$).
            The reduction is the following.

            \begin{enumerate}
                \item (setup) $\adversary'$ receives the master public key $\mpk'$ from the challenger. Then it computes $(\crs,\tpsim,\tpext)\getsr\nizkext_0(\secparam)$, the signatures keys $(\sk, \pk) \getsr \kgen(\secparam)$ and gives $\adversary$ the new master public key $\mpk = (\mpk', \pk, \crs)$.
                \item ($\ora_1$) on input $\sattr$, $\adversary'$ invokes \item ($\ora_1$) on input $\sattr$, $\adversary'$ invokes $\ora_1(\sattr)$ to have back $\ek'_\sattr$; moreover, it computes $\signature \getsr \sign(\sk, \sattr)$. Then it gives $\adversary$ the new sender key $\ek_\sattr = (\ek'_\sattr, \signature)$.
                \item ($\ora_2$) on input $\rattr$, $\adversary'$ invokes $\ora_2(\rattr)$ and forwards the received $\ek'_\rattr$.
                \item (encryption - $\ora_4$)\footnote{Only for CCA game.} invokes $\ora_4(\rattr)$ and forwards the received ciphertext $\cipher$.
                \item (challenge) on input $((\cipher', \pi), \rattr, \saccess)$, forwards the tuple $(\cipher, \rattr, \saccess)$ to the challenger.
            \end{enumerate}
        \end{proof}
    \end{lemma}
    ~\newline
    Since we have shown that Construction~\ref{constr:ame_nizk_priv} achieves CCA-privacy and preserves its authenticity, we have concluded the proof.
\end{proof}
