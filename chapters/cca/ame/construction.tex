Let $\Pi'$ be an A-ME scheme which satisfies privacy. If we use a signature scheme and a NIZK proof system, we can obtain a new A-ME scheme $\Pi$ which preserves its privacy, but also achieves authenticity. The construction is as follows.

\begin{construction}\label{constr:ame_nizk}
    Let $\Pi'=(\setup',\skgen',\rkgen', \enc',\dec')$ be an A-ME scheme, $\sig =(\kgen,\sign,\ver)$ be a signature scheme $\sig$ and $\nizk=(\init, \prover, \verifier)$ be a $f$-tSE NIZK argument for the following NP relation:

    \[
        R \eqdef \Biggl\{ \begin{array}{c}((\sattr, \signature, \raccess, \msg, r),(\mpk', \pk, \cipher))\end{array}:\begin{array}{@{}c}
            \cipher = \enc'(\ek'_\sattr, \raccess, \msg; r) \land \\
            \ver(\pk, (\sattr, \signature)) = 1
        \end{array} \Biggr\}.
    \]

    where $f(\sattr, \signature, \raccess, \msg, r) = (\sattr, \signature, \raccess, \msg)$, so $f$ extracts all the input but the randomness $r$.

    We construct the new A-ME scheme $\Pi$ as follows:
    \begin{description}
        \item[$\setup(\secparam)$:] on input the security parameter $\secparam$, the algorithm computes $(\msk', \mpk') \getsr \setup'(\secparam)$, $(\sk, \pk) \getsr \kgen(\secparam)$ and $\crs \getsr \init(\secparam)$. The algorithm outputs $\msk = (\msk', \sk)$ as the master secret key and $\mpk = (\mpk', \pk, \crs)$ as the master public key. All other algorithms are implicitly given $\mpk$ as additional input.
        \item[$\skgen(\msk, \sattr)$:] the randomized sender-key generator takes as input the master secret key $\msk = (\msk', \sk)$ and attributes $\sattr \in \bin^*$. The algorithm returns the encryption key $\ek_\sattr = (\ek'_\sattr, \signature)$, where $\ek'_\sattr \getsr \skgen'(\msk', \sattr)$ and $\signature \getsr \sign(\sk, \sattr)$.
        \item[$\rkgen(\msk, \rattr, \saccess)$:] the randomized receiver-key generator takes as input the master secret key $\msk = (\msk', \sk)$, attributes $\rattr \in \bin^*$ and a policy $\saccess:\bin^* \to \bin$. The algorithm computes the key $\dk_{\rattr, \saccess} \getsr \rkgen'(\msk', \rattr, \saccess)$.
        \item[$\enc(\ek_\sattr, \raccess, \msg)$:] the randomized encryption algorithm takes as input a secret encryption key $ek_\sattr = (\ek'_\sattr, \signature)$ for attributes $\sattr \in \bin^*$, a policy $\raccess: \bin^* \to \bin$ and a message $\msg \in \bin^*$. The algorithm first encrypts the message by computing $\cipher' \getsr \enc'(\ek'_\sattr, \raccess, \msg)$; then, it returns the ciphertext $\cipher = (\cipher', \pi)$, where $\pi \getsr \prover(\crs, (\sattr, \signature, \raccess, \msg, r),(\mpk', \pk, \cipher))$.
        \item[$\dec(\dk_{\rattr, \saccess}, \cipher)$:] the deterministic decryption algorithm takes as input a decryption key $\dk_{\rattr, \saccess}$ and a ciphertext $\cipher = (\cipher', \pi)$. The algorithm first checks whether $\verifier(\crs, (\cipher', \pk, \mpk'), \pi) = 0$. If true, returns $\bot$; else it returns $\dec'(\dk'_{\rattr, \saccess}, \cipher')$.
    \end{description}
\end{construction}

The correctness of the scheme follows directly by the correctness of the underlying A-ME scheme.

\begin{theorem}\label{theo:ame_nizk}
    Let $\Pi', \sig, \nizk$ be as above. If $\Pi'$ is CPA private, $\sig$ is EUF-CMA secure and $\nizk$ satisfies true-simulation extractability, then the A-ME scheme $\Pi$ from construction~\ref{constr:ame_nizk} is CCA-secure.

    \begin{proof}
        We can separately prove the two properties, by using Lemma~\ref{lemma:ame_auth} and Lemma~\ref{lemma:ame_priv}.

        \begin{lemma}\label{lemma:ame_auth}
            Construction~\ref{constr:ame_nizk} achieves CCA-authenticity.
            \begin{proof}
                The proof is almost the same of Lemma ~\ref{lemma:me_auth}: the only difference is that here we do not need to simulate anymore the generation of the policy keys.
            \end{proof}
        \end{lemma}

        \begin{lemma}\label{lemma:ame_priv}
            Construction~\ref{constr:ame_nizk} achieves CCA privacy.
            \begin{proof}
                As before, this proof is almost the same of Lemma ~\ref{lemma:me_priv}: the only difference is that here we do not need to simulate anymore the generation of the policy keys.
            \end{proof}
        \end{lemma}
        Since we have shown that Construction ~\ref{constr:ame_nizk} achieves both CCA-privacy and CCA-authenticity, we have concluded the proof.
    \end{proof}
\end{theorem}