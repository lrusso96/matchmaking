\begin{construction}\label{constr:ame_nizk_priv}
    Let $\Pi'=(\setup',\skgen',\rkgen', \enc',\dec')$ be an A-ME scheme, and $\nizk=(\init, \prover, \verifier)$ be a $f$-tSE NIZK argument for the following NP relation:

    \[
        R \eqdef \Biggl\{ \begin{array}{c}((\sattr, \raccess, \msg, r),(\mpk', \cipher))\end{array}:\begin{array}{@{}c}
            \cipher = \enc'(\ek'_\sattr, \raccess, \msg; r)
        \end{array} \Biggr\}.
    \]
    ~\newline\newline
    We construct the new A-ME scheme $\Pi$ as follows:
    \begin{description}
        \item[$\setup, \skgen,\enc$:] Identical to the ones in construction \ref{constr:me_nizk_priv}.
        \item[$\rkgen(\msk, \rattr, \saccess)$:] the randomized receiver-key generator takes as input the master secret key $\msk = (\msk', \sk)$, attributes $\rattr \in \bin^*$ and a policy $\saccess:\bin^* \to \bin$. The algorithm computes the key $\dk_{\rattr, \saccess} \getsr \rkgen'(\msk', \rattr, \saccess)$.
        \item[$\dec(\dk_{\rattr, \saccess}, \cipher)$:] the deterministic decryption algorithm takes as input a decryption key $\dk_{\rattr, \saccess}$ and a ciphertext $\cipher = (\cipher', \pi)$. The algorithm first checks whether $\verifier(\crs, (\cipher', \mpk'), \pi) = 0$. If true, returns $\bot$; else it returns $\dec'(\dk'_{\rattr, \saccess}, \cipher')$.
    \end{description}
\end{construction}