\section{Secret Handshakes}\label{sec:sh}

Introduced by Balfanz et al. \cite{Balfanz} in 2003, a Secret Handshake allows two members of the same group to secretly authenticate to each other and agree on a symmetric key.
During the protocol, a party can additionally specify the precise group identity (e.g., the role) that the other party should have.
\newline\newline
SH preserves the privacy of the participants, meaning that when the handshake is successful they only learn that they both belong to the same group (yet, their identities remain secret), whereas they learn nothing if the handshake fails. Subsequent work in the area focused on improving on various aspects of SH, including members' privacy and expressiveness of the matching policies (i.e., attribute-based SH).