\section{Secret Handshakes}\label{sec:sh}

Introduced by Balfanz et al. \cite{Balfanz} in 2003, a Secret Handshake is a key agreement protocol that allows two members of the same group to secretly authenticate to each other and agree on a symmetric key.
During the protocol, a party can additionally specify the precise group identity that the other party should have: e.g. it can specify its role.
\newline\newline
SH preserves the privacy of the participants: when the handshake is successful (i.e. the key is correctly derived) they only learn that they both belong to the same group; yet, their identities remain secret.
On the contrary, they learn nothing if the handshake fails.
\newline\newline
Subsequent work in the area focused on improving on various aspects of SH, including members' privacy and expressiveness of the matching policies (i.e., attribute-based SH).
