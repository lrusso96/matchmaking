\section{Notation}
We use the notation $[n] \eqdef \{1,\ldots,n\}$.
Capital boldface letters (such as $\rv{X}$) are used to denote random variables, small letters (such as $x$) to denote concrete values, calligraphic letters (such as $ \cX$) to denote sets, and serif letters (such as $\alg{A}$) to denote algorithms.
All of our algorithms are modeled as (possibly interactive) Turing machines; if algorithm $\alg{A}=(\alg{A_1},\ldots,\alg{A_k})$ has oracle access to some oracle $\alg{O}$, we often implicitly write $\mathcal{Q}_{\alg{O}}$ for the set of queries asked by $\alg{A}$ to $\alg{O}$ and $\mathcal{Q}_{\alg{O}}^i$ for the set of queries asked by $\alg{A_i}$ to $\alg{O}$.
Furthermore, we denote by $\mathcal{O}_{\alg{O}}$ (resp. $\mathcal{O}_{\alg{O}}^i$) the set of outputs returned to $\alg{A}$ (resp. $\alg{A}^i$) by $\alg{O}$.
\newline\newline
For a string $x\in\bin^*$, we let $|x|$ be its length; if $\cX$ is a set, $|\cX|$ represents the cardinality of $\cX$.
When $x$ is chosen randomly in $\cX$, we write $x \getsr \cX$. If $\alg{A}$ is an algorithm, we write $y \getsr \alg{A}(x)$ to denote a run of $\alg{A}$ on input $x$ and output $y$; if $\alg{A}$ is randomized, $y$ is a random variable and $\alg{A}(x;r)$ denotes a run of $\alg{A}$ on input $x$ and (uniform) randomness $r$.

\paragraph*{PPT.}
An algorithm $\alg{A}$ is \emph{probabilistic polynomial-time} (PPT) if $\alg{A}$ is randomized and for any input $x,r\in\bin^*$ the computation of $\alg{A}(x;r)$ terminates in a polynomial number of steps (in the input size).

\paragraph*{Negligible functions.}
Throughout the document, we denote by $\secpar\in\NN$ the security parameter and we implicitly assume that every algorithm takes as input the security parameter. A function $\nu:\NN\rightarrow[0,1]$ is called \emph{negligible} in the security parameter $\secpar$ if it vanishes faster than the inverse of any polynomial in $\secpar$, i.e.\ $\nu(\secpar)\in \bigO(1/p(\secpar))$ for all positive polynomials $p(\secpar)$.
We sometimes write $\negl$ (resp.,\ $\poly$) to denote an unspecified negligible function (resp.,\ polynomial function) in the security parameter.

\paragraph*{Indistinguishability.}
We say that $\rv{X}$ and $\rv{Y}$ are {\emph{computationally} indistinguishable, denoted $\rv{X} \cind \rv{Y}$, if for all PPT distinguishers $\dist$ we have $\CD{\dist}{X_\secpar}{Y_\secpar} \in \negl$, where
\[
    \CD{\dist}{X_\secpar}{Y_\secpar} \eqdef \left\lvert\Prob{\dist(\secparam,X_\secpar)=1} - \Prob{\dist(\secparam,\rv{Y}_\secpar)=1}\right\rvert.
\]